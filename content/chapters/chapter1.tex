\chapter{Einleitung}        % Gibt ein neues Kapitel an
\chaptermark{Einleitung}    % Ändert die Kopfzeile für das aktuelle Kapitel
\section{Aufgabenstellung}  % Gibt einen neuen Abschnitt an
Das ist das erste Kapitel.
Ein Zeilenumbruch wird automatisch entfernt, genauso    wie   überflüssige                 Leerzeichen.

Einen neuen Abschnitt beginnt man mit mindestens einer Leerzeile zwischen den Sätzen.




Dabei werden mehrere Leerzeilen wie eine behandelt.

Einen Zeilenumbruch kann man mit \\ erzwingen, das ist allerdings nach Möglichkeit zu vermeiden.

So sieht ein Literaturverweis aus: "Hier steht ein Text" \cite{OPCFoundation}.

Und so eine Abkürzung: \Gls{opcua}
\newpage
\section{Zweiter Abschnitt im ersten Kapitel}
\chaptermark{Hier steht jetzt etwas anderes.}
Eine neue Seite beginnt man mit \textbackslash newpage.
%Hier steht ein Kommentar
Kommentare werden übrigens nicht als Leerzeilen gezählt.

%So integriert man ein Bild
%\img{Bezeichnung die unter der Abbildung stehen soll}{pfad/zur/bilddatei.png}{fig:label_zur_referenzierung}
\img{Firmenlogo}{content/img/company.png}{fig:firmenlogo}

Man kann zum Beispiel wie folgt auf eine Abbildung oder Tabelle verweisen: Siehe \autoref{fig:firmenlogo}

\newpage
Hier ist Beispielhaft eine Tabelle dargestellt.

Sie kann mittels \textbackslash autoref\{tab:liste\_der\_werkstoffe\} referenziert werden.\\
Bsp: \glqq Siehe \autoref{tab:liste_der_werkstoffe}\grqq

% Beispieltabelle, & wird als Trennzeichen zwischen den Spalten verwendet
\begin{table}[htb]
    \begin{tabular}{lccc}\toprule                   % Toprule generiert die Linie über der Tabelle
    \textbf{Stoff}	&\textbf{Dichte} &\textbf{Dichte} &\textbf{chemische Bezeichnung}	\\
                & \(kg/m^{3}\)	& \(g/cm^3\) 	& \\\midrule         % Midrule generiert die Linie in der Mitte der Tabelle
    Holz		& 400...800	& 0,400...0,800	&- \\
    Plexiglas 	& 1190	& 1,19	& - 	\\
    Aluminium	& 2710	& 2,71	& Al	\\
    Titan		& 4500	& 4,5	& Ti	\\
    Gusseisen	& 7250	& 7,25	& - \\
    Stahl legiert& 7900	& 7,9	& - \\
    Messing		& 8.100...8.700	& 8,100...8,700	& Cu-Zn-Legierung\\
    Konstantan	& 8800	& 8,8	& \ch{Cu55Ni45}-Legierung\\
    Nickel		& 8910	& 8,91  & Ni \\
    Kupfer		& 8.920...8.960	& 8,920...8,960	& Cu\\
    Gold		& 19302	& 19,302& Au	\\\bottomrule               % Bottomrule generiert die Linie unter der Tabelle
    \end{tabular}
    \caption{Liste von Werkstoffen}
    \label{tab:liste_der_werkstoffe}
\end{table}